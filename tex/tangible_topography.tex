% sage_latex_guidelines.tex V1.10, 24 June 2016

\documentclass[Afour,sageh,times]{sagej}

\usepackage{moreverb,url}

\usepackage[colorlinks,bookmarksopen,bookmarksnumbered,citecolor=red,urlcolor=red]{hyperref}

\newcommand\BibTeX{{\rmfamily B\kern-.05em \textsc{i\kern-.025em b}\kern-.08em
T\kern-.1667em\lower.7ex\hbox{E}\kern-.125emX}}

% customizations
\usepackage{tabulary}
\usepackage{tabu}
\usepackage{booktabs}
\usepackage{array}
\newcommand{\ra}[1]{\renewcommand{\arraystretch}{#1}}

\def\volumeyear{2017}

\begin{document}

\runninghead{Harmon et~al.}

\title{Tangibly modeling landscapes}
%\title{Tangible Landscape}
%\title{Tangible Landscape Architectural Modeling}
%\title{Tangible Landscape Modeling}

\author{Brendan A Harmon\affilnum{1,2}, Anna Petrasova\affilnum{1}, Vaclav Petras\affilnum{1}, Helena Mitasova\affilnum{1}, and Ross Meentemeyer\affilnum{1}}

\affiliation{\affilnum{1}Center for Geospatial Analytics, North Carolina State University, USA\\
\affilnum{2}College of Design, North Carolina State University, USA}

\corrauth{Brendan A Harmon, 
Center for Geospatial Analytics,
North Carolina State University,
Raleigh, NC 27615, USA.}

\email{brendan.harmon@gmail.com}

%\begin{abstract}
%Tangible interfaces for spatial modeling
%combine embodied, kinaesthetic interaction with spatial computations. 
%Theoretically this should enable users to 
%intuitively interact 
%with multidimensional digital models of space,
%offloading challenging cognitive tasks onto the body and 
%computationally enhancing how they think about space.
%We have designed Tangible Landscape 
%-- a tangible interface powered by a geographic information system (GIS) 
%that gives 3D spatial data an interactive, physical form so that 
%users can naturally sense and shape it.
%Tangible Landscape couples a physical and a digital model of a landscape
%through real-time cycles of 
%physical manipulation, 3D scanning, spatial computation, and projected feedback.
%Through a series of 
%3D modeling experiments 
%assessed using both
%quantitative and qualitative methods 
%we determined that Tangible Landscape 
%can improve 3D spatial performance. 
%Participants produced more accurate models 
%that better represented morphological features 
%with tangible modeling than they did with either digital or analog, hand modeling.
%\end{abstract}

\begin{abstract}
We present Tangible Landscape 
-- a technology for rapidly and intuitively designing landscapes
informed by geospatial modeling, analysis, and simulation.
%
Tangible Landscape is a tangible interface powered by a geographic information system 
that gives 3D spatial data an interactive, physical form so that 
users can naturally sense and shape it.
%
It couples a physical and a digital model of a landscape
through real-time cycles of 
physical manipulation, 3D scanning, spatial computation, and projected feedback.
% 
Natural 3D sketching and real-time analytical feedback should aid
landscape architects in the design of high performance, process-based landscapes.
%
We conducted a study to assess the effectiveness of 
tangible modeling for landscape architects.
%
Landscape architecture students, academics, and professionals 
were given a series of fundamental landscape design tasks 
-- topographic modeling, cut-and-fill analysis, and water flow modeling. 
%
Their performance was assessed using both qualitative and quantitative methods.
%
With tangible modeling the participants
effectively modeled topography and water flow 
producing more accurate models 
that better represented morphological features 
than they did with either digital or analog, hand modeling.
%
When tangibly modeling
they used a rapid, iterative process informed by geospatial analytics 
that enhanced their performance.  
\end{abstract}

\keywords{Human-computer interaction, tangible interfaces, embodied cognition, geospatial modeling, topographic modeling, hydrological modeling}

\maketitle

% ---------------------------- INTRO ---------------------------- 

\section{Introduction}

% GIS IN LANDSCAPE ARCHITECTURE
Landscape architects use 
geographic information systems (GIS) to map and analyze landscapes
and computer aided design (CAD) software % 3D modeling software
to computationally represent and design landscapes.
%
While GIS can quantitatively model, analyze, simulate, and visualize 
complex spatial and temporal phenomena,
these systems can be unintuitive, challenging to use, and creatively constraining
due to the complexity of the software, 
the complex workflows, 
and the limited modes of interaction and visualization 
\cite{Ratti2004}. 
%
Due to the complex, time consuming workflows 
needed to link geospatial analysis with computer aided design 
GIS tends to play a limited, often preliminary role in the creative design process.
%
While GIS has been used extensively in landscape planning 
to model scenarios \cite{Steinitz2004,Baker2004,Steinitz2012},
it is primarily used just for preliminary research and mapping
in landscape architecture.
%
If, however, geospatial analysis, modeling, and simulation
could be seamlessly integrated into the creative design process 
then designers could rapidly develop design ideas
while quantitively testing them. 
%
Rigorously testing design concepts with quantitive measures of performance 
could drive the development of new concepts 
in a rapid, fluid creative process.
Simulation for example could inspire ideation 
linking designed form and environmental processes. 

% TANGIBLES FOR LANDSCAPE ARCHITECTURE

%The MIT Media Lab developed 


% CONCEPT / BRIEF HISTORY

% Tangibles
	% theory

% Tangible for landscape
	% Illuminating Clay
		% GIS
		% objectives

% TL

% RESEARCH QUESTIONS


% TL
	% premise: algorithmic design of landscapes
	% evolution: from illuminated clay
	% design / concept: 






%% tangibles overview
%Tangibles – tangible user interfaces – are systems that couple physical objects with digital data for more natural, embodied interaction (Dourish 2001). Tangibles give digital data an interactive, physical form and presence that users can kinaesthetically sense and manipulate. Advances in sensors, machine vision, and robotics have radically accelerated the development of tangibles – including tangibles designed for designers. Recent prototypes for tangible design include the Collaborative Design Platform (Schubert 2011, Schubert 2012) and CityScope (MIT Media Lab 2017) for urban design and Tangible Landscape (Petrasova et al. 2015), the Rapid Landscape Prototyping Machine (Robinson 2014), and Cyborg Ecologies (Responsive Artifacts and Environments Lab 2017) for landscape architecture. 
%% (ACADIA)

%%INTERACTION AND PERFORMANCE
%
%By naturally sketching in 3D, while learning from real-time analytical feedback 
%designers can work in a rapid, iterative design process 
%to quickly give their ideas form and test them.
%They can intuitively explore how form affects process. 
%With real-time geospatial simulations for example
%landscape architects can design high performance, process-based landscapes.
%With simulations seamlessly integrated into conceptual design
%physical processes like the flow of water 
%can play a generative role. 


% study of effectiveness
% claims by ishii, ratti, etc

% then
% reflect on design implications
% performative design, parametric / generative design
% performance, process-based
% physical processes as design inspiration / media
% co-evolution
% cyborg ecologies vs cthuthonic

%With computational modeling, analysis and simulation designers can generate novel forms, explore parametric variations, and quantitatively test the performance of their designs. (ACADIA)








 % CONCEPT
 %
%With this technology landscape architects and spatial scientists can collaboratively design through a seamless, rapid iterative process of intuitive ideation, geocomputational analysis, realistic rendering, and critical analysis. (ACADIA)





\clearpage



% ---------------------------- INTRO ---------------------------- 

\section{Introduction}

% ---------------------------- TOPOGRAPHIC ---------------------------- 

\section{Topographic modeling}
\subsection{Methods}
\subsection{Results}

% ---------------------------- CUT-FILL ---------------------------- 
\section{Cut-and-fill analysis}
\subsection{Methods}
\subsection{Results}

% ---------------------------- WATER FLOW ---------------------------- 
\section{Water flow modeling}
\subsection{Methods}
\subsection{Results}

% ---------------------------- DISCUSSION ---------------------------- 
\section{Discussion}

% ---------------------------- CONCLUSION ---------------------------- 

\section{Conclusion}


\clearpage

\section{Conclusion}
This study proves that tangible landscape modeling
can be an effective landscape design tool
enabling a rapid, iterative design process and
enhancing spatial performance.

% role in design process
% design process

%% concept
%Tangible Landscape -- a tangible interface for GIS -- 
%enables natural 3D sketching. % 4D spatiotemporal sketching.

% tangibles can enhance spatial performance
Through a series of experiments
we found that tangible interfaces for spatial modeling
can enhance 3D spatial performance 
in terms of speed, accuracy, and process. 
% coupling
By comparing digital, analog, and projection-augmented modeling 
we found that coupling digital and physical models as a tangible interface 
can combine the affordances of digital and analog tools
-- enabling an embodied modeling process enriched with digital data --
so that users can model intuitively, quickly, and precisely. 
Even 3D modeling experts 
 performed better with the tangible interface 
-- building more accurate models 
that better represented the morphology of the landscape --
 because they could work faster
 creating and refining details sooner.
% transfer
They were able to transfer and effectively use the
spatial skills and abilities they had developed through digital modeling
with the tangible interface.
% analytics
We also found that tangible interaction with real-time geospatial analytics
can encourage iterative modeling processes.
With the real-time difference analytic and water flow simulation  
users worked in rapid cycles of 
sculpting and digitally informed critical analysis
to build accurate models that
correctly represented the topographic and hydrologic morphology.
% process
Through this embodied process of reflection-in-action 
users were able to
observe spatial patterns, forms, and processes, 
generate and test hypotheses, 
and draw inferences. 
% cog offloading
The experiments showed that users 
were able to offload enough of the cognitive work 
of sensing and manipulating space
onto their bodies
that they could understand the
computational analytics
and adaptively re-strategize.
% future research
Further experiments are needed
to explore the role of 
spatial cognition, affect, motivation, and metacognition 
in tangible modeling.

% summary
The experiments show that Tangible Landscape,
a tangible interface for GIS, 
works as theorized and designed -- 
coupling a physical and digital model of a landscape
enables users to 
cognitively grasp topography,
intuitively shape and interact with multidimensional space, 
and offload enough cognitive work to understand 
real-time geospatial analytics. 
% process 
With Tangible Landscape users can intuitively interact with 
spatial data and scientific models using their bodies. 
% users
While novices should be able to effectively learn about 
multidimensional space and
rapidly improve their spatial abilities 
with Tangible Landscape, 
experts can effectively use it to 
rapidly develop, prototype, and test 
hypotheses about space and spatiotemporal processes.




















% ---------------------------- TEMPLATE ---------------------------- 

%\section{Template}
%\begin{enumerate}
%\item[(i)] ...
%\item[(ii)] ...
%\item[(iii)] ...
%\end{enumerate}
%
%\begin{figure*}
%%\setlength{\fboxsep}{0pt}%
%%\setlength{\fboxrule}{0pt}%
%\begin{center}
%\end{center}
%\caption{...}
%\label{F1}
%\end{figure*}
%
%Figure~\ref{F1}. You must select options for the trim/text area and
%the reference style of the journal you are submitting to.
%The choice of \verb+options+ are listed in Table~\ref{T1}.
%
%\begin{table}[h]
%\small\sf\centering
%\caption{Table.\label{T1}}
%\begin{tabular}{lll}
%\toprule
%Header 1 & Header 2 & Header 3\\
%\midrule
%\texttt{...}& ... & ....\\
%\bottomrule
%\end{tabular}\\ %[15pt] % for vspace between tabulars
%\end{table}




% ---------------------------- ENDNOTES ---------------------------- 

%\subsection{Endnotes}
%Most \textit{SAGE} journals use endnotes rather than footnotes, so any notes should be coded as \verb+\endnote{<Text>}+.
%Place the command \verb+\theendnotes+ just above the Reference section to typeset the endnotes.
%To avoid any confusion for papers that use Vancouver style references,  footnotes/endnotes should be edited into the text.


% ---------------------------- SUPPLEMENTAL MATERIAL ---------------------------- 

%\begin{verbatim}
%\begin{sm}
%To typeset a
%  "Supplemental material" section.
%\end{sm}
%\end{verbatim}

% ---------------------------- ACKNOWLEDGEMENTS---------------------------- 

% Gene Bressler

% Art Rice

% GRASS GIS Dev Community

% Blender Dev Community

%\begin{acks}
%This class file was developed by Sunrise Setting Ltd,
%Brixham, Devon, UK.\\
%Website: \url{http://www.sunrise-setting.co.uk}
%\end{acks}

% ---------------------------- BIBLIOGRAPHY ---------------------------- 

%\bibliographystyle{SageV} %Vancouver (numbered)
%\bibliography{tangible_topography.bib} 

\end{document}
