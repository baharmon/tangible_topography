\subsection{Introduction}
Theoretically tangible interfaces 
should improve spatial performance 
through embodied cognition by 
enabling natural modes of interaction, 
offloading cognitive processes onto the body, 
and computationally augmenting spatial thinking.

Theoretically tangible interfaces 
should enhance spatial thinking 
by embodying cognition so that 
interaction is natural and intuitive,
cognitive processes are offloaded onto the body,
and cognitive processes are computationally augmented.


% challenge
Spatial thinking can be challenging --
it is multidimensional and 
can involve vast amounts of data. 
% computation
Spatial analyses and simulations, however, 
can be computed efficiently in GIS. 
%
% challenge of hci


GIS can be used to quantitatively model, analyze, simulate, and visualize 
complex spatial and temporal phenomena. % spatiotemporal patterns and processes
%
While GIS can computationally enhance our understanding of space...



\subsection{Embodied spatial cognition}
% what is spatial thinking?



% definition
\citeauthor{Uttal2013} defined spatial thinking as 
`the mental processes of representing, analyzing, and drawing inferences from spatial relations' \citeyearpar{Uttal2013}. 


Psychometric tests of spatial ability -- the application of spatial thinking -- for example study spatial visualization and mental rotation \citep{Uttal2013,Uttal2013a,Ormand2014}.

We, however, do not just see space -- we also feel it; we use our bodies to feel size, shape, and volume. 
Space need not be imagined to be transformed -- haptic feedback about space informs subconscious pragmatic representations that rapidly generate action \citep{Jeannerod1997}. Spatial thinking can be embodied.

% learning about space
Spatial thinking is malleable and can be improved with training. 
The effects of spatial training can be durable -- persisting for months -- and transferable -- training in a given spatial task improves performance in other untrained spatial tasks \citep{Uttal2013}. 
% STEM
Spatial training has been shown to improve performance in science, technology, engineering, and math (STEM). 
% performance
Research, however, is needed to determine which training methods will mostly effectively improve STEM performance. 
How can spatial training integrate domain specific knowledge? And could such applied training more effectively improve performance in a given discipline? \citep{Uttal2013} 


% embodied cognition
Embodied cognition highlights the importance of a kinaesthetic, pragmatic understanding of space
and the enaction of spatial transformation -- for the act of transforming an object changes how we think about space. 
As Kirsh argues, sometimes `we know more by doing than by seeing' \citeyearpar{Kirsh2013}.

% embodiment in STEM
Embodied spatial thinking may lead to improvements in STEM performance by reducing cognitive loads with pragmatic representations and physical simulation and by enhancing perception with visual and haptic feedback. 

% embodiment in STEAM
Since spatial thinking is mediated by technology the effectiveness of training methods will depend upon their implementation, upon the technology used. 
Computer gaming has been shown to improve spatial thinking and technologies like geographic information systems (GIS) can be used to integrate domain specific knowledge \citep{Uttal2013}.
Unintuitive human-computer interaction, however, may constrain spatial thinking and add cognitive costs thus reducing the effectiveness of digitally implemented training methods. 
Embodied and computationally enriched cognition may enhance spatial thinking in novel ways
enabling and encouraging coupled creative and analytic thinking.




%%%%%%%%%%%%%%%%% GOOD TEXT
Spatial thinking can be learned; it is malleable and can be improved with experience and training. 
Spatial training has been shown to improve performance in science, technology, engineering, and math. 
It can be transferable, improving performance in other untrained spatial tasks. 
Research, however, is needed to determine which training methods will mostly effectively improve spatial thinking. 
How can spatial training integrate domain specific knowledge? And could such applied training more effectively improve performance in a given discipline? \cite{Uttal2013} 

The effectiveness of any training method will depend upon its implementation, upon the technology used. 
Computer gaming for example has been shown to improve general spatial thinking, while technologies like geographic information systems (GIS) can be used to integrate domain specific knowledge \cite{Uttal2013}.
Unintuitive human-computer interaction, however, may constrain spatial thinking and add cognitive costs that reduce the effectiveness of digitally implemented training methods. 
%%%%%%%%%%%%%%%%%%%%%%%%%


domain specific knowledge
spatial thinking
% Technologies like geographic information systems (GIS) can be used to integrate domain specific knowledge \cite{Uttal2013}.



% geography
it is also used at a geographic scale in disciplines like 
architecture to create forms, 
geography and ecology to study patterns and processes in the environment, 
urban planning to organize cities,
and 
engineering to design structures and systems. % citations for each example

% importance in science, technology, education, and math

Spatial thinking is also used extensively in science, technology, engineering, the arts, and math. 
Physicists use it to simulate physical processes,
biologists and chemists use it to map and manipulate molecules,
electrical engineers use it to design circuits, 
architects use it to design buildings, 
sculptors use it to shape forms, 
and mathematicians use it to study topology. \\

Spatial relationships, patterns, forms, and processes can be computationally modeled, analyzed, simulated, and represented. 

Space is often modeled, analyzed, simulated, and represented computationally at geographic scales
due of the sheer quantity and complexity of data. 
Computational models can process large sets of spatial data and model or simulate complex spatiotemporal processes.


Due to their complexity spatial relationships, patterns, forms, and processes are often computationally modeled, analyzed, simulated, and represented. 

% citation for each, examples (CAD/BIM, GIS, remote sensing, global datasets)
Typically the results of computational spatial models are presented graphically using a graphical user interface (GUI) and require sophisticated spatial thinking to parse and understand. 

%
\subsection{Spatial technology}

Digital technologies like computer-aided design (CAD) software, 3D modeling and animation software, and geographic information systems (GIS)

Computer-aided design (CAD) software, 3D modeling and animation software, and geographic information systems (GIS) for example are used to model space

% What is GIS?
Geographic information systems computationally store, model, analyze, simulate, and represent geospatial patterns and processes. 
%
The open source project GRASS GIS for example supports 
`geospatial data management and analysis, image processing, graphics and maps production, spatial modeling, and visualization.'

% Unintuitive human-computer interaction, however, may constrain spatial thinking and add cognitive costs. 
% that reduce effectiveness

... presented graphically using a graphical user interface (GUI) and require sophisticated spatial thinking to parse and understand. 
% citations and examples: mental rotation, stem learning
Furthermore it can be challenging to interact with these computational models using a GUI 
due to the high cognitive load of visualizing multidimensional space (and time) % citation
% split sentence
and the disconnect between intention, action, and feedback -- 
the disjunction between ones idea, it expression constrained by the input device (such as a mouse and keyboard, a touch screen, or a digitizing pen), and the graphical representation \cite{Dourish2001,Ishii2008}. 







%%%%%%%%%%%%%%%%%%%%%%%%%%%%%%%%%%%%%%%%%%%%%%%%%%%%%



Tangible interfaces -- interfaces that couple physical and digital data for intuitive interaction \citep{Dourish2001} -- 
are designed to by physically manifesting digital data so that we can cognitively grasp and absorb it,
so that we can think with it rather than about it \citep{Kirsh2013}. 


Tangible computing aims to embody computing 
by coupling physical and digital data \citep{Dourish2001} -- 
by physically manifesting digital data so that we can cognitively grasp and absorb it,
so that we can think with it rather than about it \citep{Kirsh2013}. 

\citeauthor{Ishii1997} envisioned that TUIs would  `take advantage of natural physical affordances to achieve a heightened legibility and seamlessness of interaction between people and information' \citeyearpar{Ishii1997}. 



The theses
that cognition is embodied and situated,
that we sometimes understand more through acting than by watching,
that tools mediate how we perceive, think, and act, 
and that tools like computers can extend our capacity to think \citep{Kirsh2013}
have inspired a new paradigm of human-computer interaction -- embodied interaction.  



Given that computers can functionally expand our capacity to think \citep{Kirsh2013}
...



% Spatial thinking

Spatial thinking -- `the mental processes of representing, analyzing, and drawing inferences from spatial relations' \cite{Uttal2013} -- 
is used pervasively in everyday life at a personal scale for tasks like recognizing things, manipulating things, and way-finding; 
it is also used at a geographic scale in disciplines like 
art and architecture to create forms, 
geography and ecology to study patterns and processes in the environment, 
urban planning to structure cities,
and 
engineering to design structures and systems








% Spatial thinking

Spatial thinking -- `the mental processes of representing, analyzing, and drawing inferences from spatial relations' \cite{Uttal2013} -- is used in diverse disciplines to study 



to
study environmental patterns and processes
creatively design form and function
engineer structures and systems






% Tangible Landscape
We have designed...

Imagine being able to hold a GIS in your hands, feeling the shape of the earth, sculpting its topography, and directing the flow of water. We present Tangible Landscape, an open source tangible interface powered by GRASS GIS. Tangible Landscape physically, interactively manifests geospatial data so that you can naturally feel it, see it, and shape it. This makes GIS far more intuitive and accessible for beginners, empowers geospatial experts, and creates new exciting opportunities for developers - like gaming with GIS. 


\section{...}

% Theory

% Embodied cognition

Cognition can be embodied -- it can be embedded in the body and based on bodily experience. 
Higher cognitive processes, the traditional realm of cognitive science, 
can rely on lower level processes such as emotion and sensorimotor processes that link perception and action. 
Thus feeling, action, and perception can be functionally integral to thought \cite{Hardy-Vallee2008}. 
We can for example physically simulate cognitive processes, offloading cognition onto action to functionally `think with our bodies' \cite{Kirsh2013}. 
We can cognitively grasp objects, temporarily, contingently incorporating tools into our body schema \cite{Kirsh2013}.


% Spatial cognition
Many conceptions and studies of spatial thinking focus on a visual, semantic understanding of space. 
Embodied cognition, however, highlights the importance of a kinaesthetic, pragmatic understanding of space
and the enaction of spatial transformation -- for the act of transforming an object changes how we think about space. 
As \cite{Kirsh2013} argues, sometimes `we know more by doing than by seeing' \cite{Kirsh2013}.


\cite{Uttal2013} defined spatial thinking as 
`the mental processes of representing, analyzing, and drawing inferences from spatial relations' \cite{Uttal2013}. 
This definition is based on a semantic rather than pragmatic understanding of space, an understanding based on visual rather haptic, kinaesthetic feedback. 
In this paradigm space is imagined rather than felt and spatial transformation is imagined rather than enacted. 
Psychometric tests of spatial ability -- the application of spatial thinking -- for example study spatial visualization and mental rotation \cite{Uttal2013,Uttal2013a,Ormand2014}.
We, however, do not just see space -- we also feel it; we use our bodies to feel size, shape, and volume. 
Space need not be imagined to be transformed -- haptic feedback about space informs subconscious pragmatic representations that rapidly generate action \cite{Jeannerod1997}. Spatial thinking can be embodied.
%%% Definition: spatial cognition

%%% learning about space
Spatial thinking is malleable and can be improved with training. 
The effects of spatial training can be durable -- persisting for months -- and transferable -- training in a given spatial task improves performance in other untrained spatial tasks \cite{Uttal2013}. 
How can spatial training integrate domain specific knowledge? And could such applied training more effectively improve performance in a given discipline? \cite{Uttal2013} 

Embodied spatial thinking may lead to improvements in performance by reducing cognitive loads with pragmatic representations and physical simulation and by enhancing perception with visual and haptic feedback. 

Since spatial thinking is mediated by technology the effectiveness of training methods will depend upon their implementation, upon the technology used. 
Computer gaming has been shown to improve spatial thinking and technologies like geographic information systems (GIS) can be used to integrate domain specific knowledge \cite{Uttal2013}.
Unintuitive human-computer interaction, however, may constrain spatial thinking and add cognitive costs thus reducing the effectiveness of digitally implemented training methods. 
Embodied and computationally enriched cognition may enhance spatial thinking in novel ways
enabling and encouraging coupled creative and analytic thinking.


% Computational spatial modeling

% Digital disembodiment

% Tangible interfaces (to embody spatial cognition) (affordance, feedback, intuition)
Tangible computing aims to embody computing 
by coupling physical and digital data \cite{Dourish2001} -- 
by physically manifesting digital data so that we can cognitively grasp and absorb it,
so that we can think with it rather than about it \cite{Kirsh2013}. 
\cite{Ishii1997} envisioned that TUIs would  `take advantage of natural physical affordances to achieve a heightened legibility and seamlessness of interaction between people and information' \cite{Ishii1997}. 

To understand how computing transforms cognition we need to study
`the complex coordination between external and internal simulation, between doing things internally and doing things externally' \cite{Kirsh2013}. % Research question!

% A brief history of tangible computing?

Theoretically tangible interaction should offload cognitive processes through bodily action, physical simulation, and digital computation.

Should improve spatial performance.

Research questions: 
Can tangible interfaces improve spatial performance?
Which tangible analytics improve spatial performance the most?

Aim: Improve spatial performance

A comparative study of 3D spatial performance with hand modeling, digital modeling, and tangible interaction.

Two experiments. 




%%%%%%%%%%%%%%%%%













% tangibles and embodiment

Theoretically tangible interfaces that physically manifest digital data should enable users to cognitively grasp the data as an extension of their bodies so that they can automatically, immediately, and subconsciously interact with it. 


When thinking with the body, when embodying cognition, one can physically simulate cognitive processes and cognitively grasp objects, temporarily, contingently incorporating tools into our body schema \cite{Kirsh2013}.




% tui


Cognition can be embodied -- it can be embedded in the body and based on bodily experience. 
Higher cognitive processes, the traditional realm of cognitive science, 
can rely on lower level processes such as emotion and sensorimotor processes that link perception and action. 
Thus feeling, action, and perception can be functionally integral to thought \cite{Hardy-Vallee2008}. 
We can for example physically simulate cognitive processes, offloading cognition onto action to functionally `think with our bodies' \cite{Kirsh2013}. 
We can cognitively grasp objects, temporarily, contingently incorporating tools into our body schema \cite{Kirsh2013}.

Tangible interfaces should allow users think about digital space with their bodies. 


By enabling users to kinaesthetically understand and shape digital space


We designed Tangible Landscape so that spatial cognition can be embodied -- embedded in the body and based on bodily experience -- so that users can kinaesthetically understand and shape space

By making GIS more intuitive and accessible for beginners, empowers geospatial experts, and creates new exciting opportunities for developers - like gaming with GIS. 


TUIs make pragmatic representations of digital data. 
Pragmatic representations are conceptual models for rapidly generating action
that are primarily based on tactile feedback and are processed automatically, immediately, 
and subconsciously \cite{Jeannerod1997}. 
%

By physically manifesting data so that we can pragmatically
interact with computations,  
TUIs should enable rapid, intuitive action and expression 
in a way that was not possible in visual computing. 

By enabling embodied cognition in human-computer interaction 
TUIs should let us cognitively grip data as an extension of our bodies, 
intuitively manipulate data, and physically simulate processes. 

Tangible computing aims to embody computing 
by coupling physical and digital data \cite{Dourish2001} -- 
by physically manifesting digital data so that we can cognitively grasp and absorb it,
so that we can think with it rather than about it \cite{Kirsh2013}. 
\cite{Ishii1997} envisioned that TUIs would  `take advantage of natural physical affordances to achieve a heightened legibility and seamlessness of interaction between people and information' \cite{Ishii1997}. 



