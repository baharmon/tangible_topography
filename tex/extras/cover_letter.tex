% !TEX encoding = UTF-8 Unicode

% sage_latex_guidelines.tex V1.10, 24 June 2016

\documentclass[Afour,sagev,times]{sagej}

% customizations
\usepackage[super]{nth}
\usepackage[inline]{enumitem}
\usepackage{moreenum}
\usepackage{tabulary}
\usepackage{tabu}
\usepackage{booktabs}
\usepackage{array}
\usepackage[super]{nth}
\usepackage{listings}
\usepackage{float}
\usepackage{tikz}
\usepackage{upquote}
\usepackage{graphicx}
\usepackage{epstopdf}
\usepackage{tabularx}
\usepackage{import} 
\newcolumntype{Y}{>{\raggedright\arraybackslash}X}
\newcommand{\ra}[1]{\renewcommand{\arraystretch}{#1}}

% special characters
\usepackage{gensymb}
\usepackage{amssymb}
\usepackage{pifont}
\newcommand{\cmark}{\ding{51}}
\newcommand{\xmark}{\ding{55}}

% sage packages and commands
\usepackage{moreverb}
\usepackage[hyphens]{url}
\usepackage[colorlinks,bookmarksopen,bookmarksnumbered,citecolor=red,urlcolor=red]{hyperref}
\newcommand\BibTeX{{\rmfamily B\kern-.05em \textsc{i\kern-.025em b}\kern-.08em
T\kern-.1667em\lower.7ex\hbox{E}\kern-.125emX}}


% ---------------------------- METADATA ---------------------------- 

\def\volumeyear{2017}
\begin{document}
\runninghead{Harmon et~al.}
\title{Tangible topographic modeling for landscape architects}
\author{Brendan Harmon\affilnum{1,2}, Anna Petrasova\affilnum{2}, Vaclav Petras\affilnum{2}, Helena Mitasova\affilnum{2}, and Ross Meentemeyer\affilnum{2}}
\affiliation{\affilnum{1}Robert Reich School of Landscape Architecture, Louisiana State University, USA\\
\affilnum{2}Center for Geospatial Analytics, North Carolina State University, USA}
\corrauth{Brendan A Harmon, 
Robert Reich School of Landscape Architecture,
Louisiana State University,
Baton Rouge, LA 70803, USA.}
\email{brendan.harmon@gmail.com}

% ---------------------------- BODY ---------------------------- 

\maketitle

\section{Cover letter}

Dear Editors,

I would like to submit this paper 
to SAGE's International Journal of Architectural Computing. 
%
This paper reviews existing research and design work about 
tangible interfaces for landscape architecture,  
introduces our tangible interface -- Tangible Landscape --
and describes a series of laboratory-based user experiments
using quantitative and qualitative methods
to assess 3D spatial performance for key landscape design tasks.

\paragraph{Relationship to similar publications about Tangible Landscape}

% technology
This paper introduces
for the third generation of Tangible Landscape,
a tangible interface for geospatial modeling. 
The first generation of this system was described in the paper 
\emph{GIS-based environmental modeling with tangible interaction and dynamic visualization} \cite{Petrasova2014}
and the second generation of the system was described in the book
\emph{Tangible Modeling with Open Source GIS}
\cite{Petrasova2015}. 
The coupling of Tangible Landscape with VR
has already been described in a demo paper titled
\emph{Immersive Tangible Geospatial Modeling}
that has been published in ACM SIGSPATIAL 2016 \cite{Tabrizian2016}
and in the conference paper 
\emph{Tangible Immersion for Ecological Design}
accepted for ACADIA 2017 \cite{Tabrizian2017}. 
These publications described the design of the systems, 
the technical implementation and the underlying algorithms, 
and detailed case studies, but did not include user experiments.
%
This paper has the first complete user study for Tangible Landscape.

\paragraph{Relationship to similar publications about user experiments}

The user study described in this paper 
addresses important basic research questions 
about 3D modeling and tangible interaction
in landscape architecture.
%
The experiments used novel methods such as geospatial modeling 
to spatially analyze and quantitatively assess 3D spatial performance. 
%
Two of the three user experiments described in this research 
are based on pilot studies 
--
\emph{Embodied Spatial Thinking in Tangible Computing}
\cite{Harmon2016b}
and
\emph{Tangible Landscape: cognitively grasping the flow of water}
\cite{Harmon2016c}
--
with fewer participants, less sophisticated methods, and only preliminary findings.
% 
The methods and results described in this paper 
differ substantially from these pilot studies. 
New methods include statistical transformations, 
cellular statistics, morphological analysis, and 3D visualization. 
%
This research also includes qualitative methods 
that were not used in the pilot studies.

% ---------------------------- BIBLIOGRAPHY ---------------------------- 

\bibliographystyle{SageV}
\bibliography{../tangible_topography.bib}

\end{document}



