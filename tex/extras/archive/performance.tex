

% computation
While parametric modeling informed by computational analysis 
can play a generative, form-finding role 
in performative architecture, 
performance-based landscape architecture 
currently relies largely upon research and assessment.
Since landscapes are shaped by physical and ecological processes
geospatial modeling and simulation could play a truly generative role 
in the design of high performance landscapes.
%
The challenge is seamlessly integrating 
geospatial modeling and simulation
into the creative design process.


%%%%%%%%%%%%%%%%%%%%%%

% performance
Performance-based design is an emerging paradigm 
in both architecture and landscape architecture. 
% architectural performance
In performative architecture 
the functional and cultural performance of the building 
-- quantified through factors such as
structure, energy use, acoustics, temperature, and wind --
drives the design.
\cite{Kolarevic2005}
% landscape performance
Similarly in performance-based landscape architecture 
the functional performance of the landscape
-- quantified through factors such as
ecosystem services, plant health, soils, 
hydrology, geomorphology, and biodiversity -- 
drives the design. 
% strategies
Strategies for performance-based landscape architecture include
monitoring and assessing how well landscapes function
after they have been built, \cite{Yang2016}
analyzing how landscapes function before designing interventions,
testing alternatives through adaptive experiments,
adaptively managing landscapes as they change, 
and comparing alternative future scenarios 
based on quantitative metrics. \cite{Lovell2015}
%%%%%%%%%%%%%%%%%%%%%%

% architectural functional performance
example: structural geometry of Grimshaw's Eden Project biomes
or Great Court at the British Museum by Foster + Partners

% performance as performance art
examples: Milwaukee Art Museum, BIX / Kunsthaus Graz, Blur Building

%%%%%%%%%%%%%%%%%%%%%%

% intro
....

% evidence-based design
While evidence-based landscape architecture
is focused on quantitatively assessing 
environmental, economic, and social 
measures of performance
for built landscapes
\cite{Yang2016}, 
other strategies
for performance-based landscape architecture include
analyzing how landscapes function before designing,
adaptively experimenting to test how well designs will work,
adaptively managing landscapes as they change, 
and comparing alternative future scenarios based on quantitative metrics
\cite{Lovell2015}. 

%%%%%%%%%%%%%%%%%%%%%%

In performance-based landscape architecture
landscape architects may
analyze how a landscape functions before designing,
monitor and assess how well a landscape functions after it has been built,
adaptively experiment to test how well a design will work,
adaptively manage a landscape as it changes,
compare alternative scenarios \cite{Lovell2015}.


%%%%%%%%%%%%%%%%%%%%%%

Strategies for performance-based landscape architecture include
monitoring and assessing how well landscapes function
after they have been built \cite{Yang2016},
analyzing how landscapes function before designing,
adaptively experimenting to test how well designs will perform,
adaptively managing landscapes as they change, 
and comparing alternative future scenarios 
based on quantitative metrics \cite{Lovell2015}.


%%%%%%%%%%%%%%%%%%%%%%

% landscape performance
In performance-based landscape architecture
landscape architects may
analyze how a landscape functions before designing,
monitor and assess how well a landscape functions 
after it has been built \cite{Yang2016},
adaptively experiment to test how well a design will work,
adaptively manage a landscape as it changes,
compare alternative scenarios %, 
or design for change and evolution driven by natural processes.

%%%%%%%%%%%%%%%%%%%%


% `stormwater quality, stormwater quantity, vegetation, soil and social performance of the site'

%%is currently focused on assessment -- 
%%on the assessment hydrology, plant health, soils, programmatic use, etc.~
%%after construction. 

% adaptive management
\cite{Salafsky2001}

% strategies
% comparative analysis
	% analyze the performance of landscape
	% GIS-based analysis
% adaptive experimentation
	% monitoring
	% adaptive management
% scenario planning / alternative futures
	% metrics for performance like ecosystem services
\cite{Lovell2015}

% autopoesis
% corner

