In engineering, design, and the arts 
computer-aided design (CAD) and 3D modeling software are used 
to interactively, computationally model, analyze, and animate complex 3D forms.
%
In scientific computing
multidimensional spatial patterns and processes can be 
mathematically modeled, simulated, and optimized 
using geographic information systems (GIS), 
geospatial programming, and 
spatial statistics. 
%
GIS can be used to quantitatively model, analyze, simulate, and visualize 
complex spatial and temporal phenomena 
-- computationally enhancing users' understanding of space. 
%
With extensive libraries for point cloud processing, 
3D vector modeling, and
surface and volumetric modeling and analysis
GIS are powerful tools for studying 3D space.

GIS, however, can be unintuitive, challenging to use, and creatively constraining
due to the complexity of the software, 
the complex workflows, 
and the limited modes of interaction and visualization 
\cite{Ratti2004}. 
%
Unintuitive interactions with GIS can 
frustrate users,
constrain how they think about space, and add new cognitive burdens
that require highly developed spatial skills and reasoning to overcome. 
%
The paradigmatic modes for interacting with GIS today 
-- command line interfaces (CLI) and
graphical user interfaces (GUI) -- 
require physical input into devices like 
keyboards, mice, digitizing pens, and touch screens, 
but output data visually as text or graphics. 
%
Theoretically this disconnect between intention, action, and feedback 
makes graphical interaction unintuitive \cite{Dourish2001,Ishii2008}. 
%
Since users can only think about space visually with GUIs,
they need sophisticated spatial abilities 
like mental rotation \cite{Shepard1971,Just1985}
%spatial visualization, and spatial perception \cite{Linn1985}
to parse and understand, much less manipulate 
3D space. %in a GIS using a GUI. 

In order to make 3D GIS more natural and intuitive to use
we have designed Tangible Landscape 
-- a tangible interface for GIS --
that physically manifests 3D data 
so that users can feel and manipulate it with their bodies 
(Fig.~\ref{fig:tl_flow}). 
%
Our goal is for users with little or no computer experience 
to be able to intuitively, collaboratively explore 
%higher dimensional %multidimensional 
3D spatial data 
and interact with scientific models
so that they can 
rapidly test ideas while learning from computational feedback. 





% THEORY
\subsection{Tangible, embodied interaction}

In embodied cognition higher cognitive processes are 
grounded in, built upon, and mediated by bodily experiences 
such as kinaesthetic perception and action \cite{Hardy-Vallee2008}. 
%
Tangible interfaces 
-- interfaces that couple physical and digital data \cite{Dourish2001} -- 
are designed to enable embodied interaction
by physically manifesting digital data 
so that users can cognitively grasp and absorb it,
thinking with it rather than about it \cite{Kirsh2013}. 
%
Embodied interaction should be highly intuitive --
drawing on existing motor schemas
and seamlessly connecting intention, action, and feedback.
%
It should reduce users' cognitive load 
by enabling them to
physically simulate processes 
and offload tasks like 
spatial perception and manipulation onto the body
\cite{Kirsh2013}.

With GIS users can computationally offload complex cognitive tasks 
like analyzing spatial patterns and simulating spatiotemporal processes.
Tangible interfaces for geospatial modeling should, 
therefore, enhance users' spatial performance 
-- their ability to sense, manipulate, and interact with multidimensional space -- 
for challenging tasks 
like sculpting topography and guiding the flow of water
by combining these physical and computational affordances.