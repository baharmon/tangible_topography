% !TEX encoding = UTF-8 Unicode

% Compilation using 'acmsmall.cls' - version 1.3 (March 2012), Aptara Inc.
% (c) 2010 Association for Computing Machinery (ACM)
%
% Questions/Suggestions/Feedback should be addressed to => "acmtexsupport@aptaracorp.com".
% Users can also go through the FAQs available on the journal's submission webpage.
%
% Steps to compile: latex, bibtex, latex, latex


\documentclass[prodmode,acmtochi]{acmsmall} % Aptara syntax

% Package to generate and customize Algorithm as per ACM style
\usepackage[ruled]{algorithm2e}
\renewcommand{\algorithmcfname}{ALGORITHM}
\SetAlFnt{\small}
\SetAlCapFnt{\small}
\SetAlCapNameFnt{\small}
\SetAlCapHSkip{0pt}
\IncMargin{-\parindent}

% Packages
\usepackage[super]{nth}
\usepackage[inline]{enumitem}
\usepackage{moreenum}
\usepackage{longtable}
\usepackage{tabu}
\usepackage{booktabs}
\usepackage{array}
\usepackage[super]{nth}
\usepackage{listings}
\usepackage{float}
\usepackage{minted}
\usemintedstyle{bw}
\newcommand{\ra}[1]{\renewcommand{\arraystretch}{#1}}
\usepackage{tabulary}

% Special characters
\usepackage{amssymb}% http://ctan.org/pkg/amssymb
\usepackage{pifont}% http://ctan.org/pkg/pifont
\newcommand{\cmark}{\ding{51}}%
\newcommand{\xmark}{\ding{55}}%

% Metadata Information
\acmVolume{0}
\acmNumber{0}
\acmArticle{0}
\acmYear{2016}
\acmMonth{0}

% Copyright
%\setcopyright{acmcopyright}
%\setcopyright{acmlicensed}
%\setcopyright{rightsretained}
%\setcopyright{usgov}
%\setcopyright{usgovmixed}
%\setcopyright{cagov}
%\setcopyright{cagovmixed}

% DOI
\doi{0000001.0000001}

%ISSN
\issn{1234-56789}

% Document starts
\begin{document}

% Page heads
\markboth{B. Harmon et al.}{Cognitively Grasping Topography with Tangible Landscape}

% Title portion
\title{Cognitively Grasping Topography with Tangible Landscape} 
\author{BRENDAN ALEXANDER HARMON
\affil{North Carolina State University}
ANNA PETRASOVA
\affil{North Carolina State University}
VACLAV PETRAS
\affil{North Carolina State University}
HELENA MITASOVA
\affil{North Carolina State University}
ROSS ​KENDALL MEENTEMEYER
\affil{North Carolina State University}
EUGENE BRESSLER
\affil{North Carolina State University}
ART RICE
\affil{North Carolina State University}}

\begin{abstract}
%
\end{abstract}

%
% The code below should be generated by the tool at
% http://dl.acm.org/ccs.cfm
% Please copy and paste the code instead of the example below. 
%
\begin{CCSXML}
<ccs2012>
<concept>
<concept_id>10003120.10003121</concept_id>
<concept_desc>Human-centered computing~Human computer interaction (HCI)</concept_desc>
<concept_significance>500</concept_significance>
</concept>
<concept>
<concept_id>10003120.10003121.10003122.10011749</concept_id>
<concept_desc>Human-centered computing~Laboratory experiments</concept_desc>
<concept_significance>500</concept_significance>
</concept>
</ccs2012>
\end{CCSXML}

\ccsdesc[500]{Human-centered computing~Human computer interaction (HCI)}
\ccsdesc[500]{Human-centered computing~Laboratory experiments}
%
% End generated code
%

\keywords{Human-computer interaction, tangible interfaces, interaction design, physical computation, embodied cognition, spatial thinking, geospatial modeling}

\acmformat{Brendan A. Harmon, Anna Petrasova, Vaclav Petras, Helena Mitasova, Ross K. Meentemeyer, Eugene H. Bressler, and Art Rice, 2016. Embodied Spatial Cognition in Tangible Computing.}

\begin{bottomstuff}
Author's addresses: B. A. Harmon {and} A. Petrasova {and} V. Petras {and} H. Mitasova {and} R. K. Meentemeyer, Center for Geospatial Analytics, North Carolina State University; B. A. Harmon, E. H. Bressler {and} A. Rice, Department of Landscape Architecture, North Carolina State University.
\end{bottomstuff}

%\maketitle
%\pagebreak

\section{Semi-structured interview guidelines}
\vspace*{0.5em}

\subsection{Aim}
Understand how tangible interfaces for geospatial modeling change how users model.
\vspace*{0.5em}

\subsection{Interview goals}
\begin{itemize}
\item Map participants' analog, hand modeling processes
\item Map participants' digital modeling processes
\item Map participants' augmented modeling processes
\item Map participants' tangible modeling processes with the difference analytic
\item Map participants' tangible modeling processes with the water flow analytic
\end{itemize}
\vspace*{0.5em}

\subsection{\emph{Topic:} Modeling process}
\begin{itemize}
\item Please describe your modeling process with each technology
\item Did you work additively or subtractively? A mix?
\item Did you work in a linear or iterative, exploratory process?
\item How did this technology aid you? What did it let you to do?
\item Did this technology constrain you in any way?
\end{itemize}
\vspace*{0.5em}

\subsection{\emph{Topic:} Intuition}
\begin{itemize}
\item How intuitive was it? 
\item Could you model what you intended?
\item Did you have to think about how to modeling? Or could you just act?
\end{itemize}
\vspace*{0.5em}

\subsection{\emph{Topic:} Metacognition}
\begin{itemize}
\item We asked you to sculpt a model of the study landscape. Please describe your thought process while sculpting. 
\item Did you strategize about how to model? If so what was your modeling strategy? 
\item Did your modeling strategy evolve as you worked?
\end{itemize}
\vspace*{0.5em}

\subsection{Topic: perception and experience}
\begin{itemize}
\item How did it feel to sculpt a 3D model with this technology?
\item Was it stressful? Was it fun?
\item Did the technology change how you perceived distance, depth, form, or volume?  
\end{itemize}

\vfill

\section{Interview notes}
%
\subsection{Interview I}
%
\paragraph{Digital modeling}
Understanding how Rhino works 
-- i.e.~the underlying mathematical representation, NURBS -- is very important. 
Once you understand that is a tension field 
then you understand how to shape it, how to make it do what you want.
%
\paragraph{Tangible modeling}
We all already understand how sand works. 
We understand sand, but not necessarily these analytics 
-- the difference analytic or the water flow simulation.

\subsection{Interview II}
\paragraph{Digital modeling}
Working in Vue is like modeling in wax.
Working with multiple tools in Vue gave me more control, more options
than Rhino's gumball did. It is easier than Rhino.
Even though it is easy it is still very important to learn the tools.
I was exploring what the tools could do.
%
\paragraph{Analog, hand modeling}
I worked additively, then subtractively, smoothing.
The sculpting tool gave sharpness --
the sharp edge let me smooth in a way my fingers couldn't.
It felt like drawing or laying concrete.
Feeling is important -- I could feel subtle changes in topography.
%
\paragraph{Projection augmented modeling}
I worked additively. 
I sculpted with my fingers rather than the wooden modeling tool. 
The projection helped to orient me  
by defining features like the shoreline.
%
\paragraph{Tangible modeling}
The water flow analytic reminds of me of playing in creeks and grading streams as a kid. 

\subsection{Interview III}
%
\paragraph{Digital modeling}
Digital modeling has a long learning curve. 
It was not intuitive. I was quite anxious. 
The interaction was quite abstract, quite indirect 
-- I was moving points to change the surface.
Working with the surface was like draping a fabric.
%
\paragraph{Analog, hand modeling}
My sense of touch took away the mystery of topography.
I could sculpt like reading braille. I could feel the shape with my fingers.
It was intuitive and calming. 
%
\paragraph{Projection augmented modeling}
With the projected maps draped over the sand
there were layers of systems, strata overlaid.
%
\paragraph{Tangible modeling}
The difference analytic was the best. 
I tried to make it match. 
I was constantly rebuilding to make it match.
The water flow analytic was useful for thinking about form, 
about what form does --  why water flows where it does.
Seeing the flow takes away the mystery of topography.
Our students tend to have a linear design process.
Because Tangible Landscape gives immediate results 
it encourages an iterative process.

\subsection{Interview IV}
%
\paragraph{Digital modeling}
My strategy was to model the outside borders first. 
Then the interior. 
%
\paragraph{Analog, hand modeling}
I have done lots of sculpture so
I knew how to feel the shape of the model. 
And the desk lamp cast shadows so I could visually perceive depth.
%
\paragraph{Projection augmented modeling}
The contours were just a guide. 
My general strategy was additive. 
I felt with my hands to try to match the contours. 
If I saw concavity in the contours 
then I felt the sand and sculpted that concavity.
Finding the relative height, however, was challenging -- it was subtle.
Most people don't understand contours. 
They have to be taught. 

\subsection{Interview IV}
%
\paragraph{Tangible modeling}
Tangible Landscape let me tinker. 
I could rapidly create, making new iterations. 
I could try something, see and feel it -- directly experience it --
and try again. Reinvent it.
Tinkering like this is a learning process. Learning through doing.
Tangible Landscape lowers the stakes so that you're not too invested.
You're ready to fail. So you can intuitively explore, 
while reflecting on what you've done,
what you're doing.

\vfill

% Bibliography
\bibliographystyle{ACM-Reference-Format-Journals}
\bibliography{tangible_topography.bib}

\end{document}



