\begin{table}
\caption{Interviews}
%\ra{1.3}
\ra{2}
\begin{tabular}{p{0.2\textwidth} p{ 0.75\textwidth}}
\toprule
Technology & Select comments\\
\midrule
%
Digital
%
& Rhino has a long learning curve.\\
& Understanding how Rhino works 
-- i.e.~the underlying mathematical representation, NURBS -- is very important. 
Once you understand that is a tension field 
then you understand how to shape it, how to make it do what you want.\\
%
Analog 
%
& I worked additively, then subtractively, smoothing.
The sculpting tool gave sharpness --
the sharp edge let me smooth in a way my fingers couldn't.
It felt like drawing or laying concrete.
Feeling is important -- I could feel subtle changes in topography.\\
& My sense of touch helped me to understand the topography.
I could sculpt like reading braille. I could feel the shape with my fingers.
It was intuitive and calming.\\
& I have done lots of sculpture so
I knew how to feel the shape of the model. 
And the desk lamp cast shadows so I could visually perceive depth.\\
%
Tangible Landscape
% 
& We all already understand how sand works. 
We understand sand, but not necessarily these analytics 
-- the difference analytic or the water flow simulation.\\
& The water flow simulation reminds of me of playing 
in creeks and grading streams as a kid.\\
& The difference analytic was the best. 
I tried to make it match. 
I was constantly rebuilding to make it match.
The water flow simulation was useful for thinking about form, 
about what form does --  why water flows where it does.
Seeing the flow takes away the mystery of topography.
Our students tend to have a linear design process.
Because Tangible Landscape gives immediate results 
it encourages an iterative process. \\
& Tangible Landscape let me tinker. 
I could rapidly create, making new iterations. 
I could try something, see and feel it -- directly experience it --
and try again. Reinvent it.
Tinkering like this is a learning process. Learning through doing.
Tangible Landscape lowers the stakes so that you're not too invested.
You're ready to fail. So you can intuitively explore, 
while reflecting on what you've done,
what you're doing.\\
%
\bottomrule
\end{tabular}
\label{table:interviews}
\end{table}