

We did not use psychometric tests
to assess spatial ability because
these do not address 
geographic scales,
spatial relations
domain-specific knowledge and abilities 
\cite{Lee2009,Bednarz2011,Wakabayashi2011},
or embodied cognitive ability. 
%

standardized test of spatial thinking abilities

While there is a standardized test of spatial thinking abilities \cite{Bednarz2011,Lee2012},
we did not use this because
visually based, while
we are interested in tangibles and embodiment 

Therefore we developed a methodology
for quantitatively testing
spatial performance
using spatial analytics


accounts for 
spatial form, relationships, and processes

addresses the end product
applicable for visual and embodied spatial thinking

could be extended to account for 
spatiotemporal processes

and participant's performance over time


Psychometric, pen-and-pencil tests have been used successfully by researchers to assess subjects’
visualization and orientation abilities at a table-top scale [6,7]. These tests left geographers, earth scientists, and environmental scientists dissatisfied because they referred to a scale that was not the most relevant for their disciplines and because they did not test spatial relations, a dimension important to understanding

One goal of the present studywas to develop a standard-
ized test of spatial thinking abilities (the spatial thinking ability test (STAT)) that integrates geography content knowledge and spatial skills. Currently no standardized instrument for assessing the set of spatial thinking skills discussed previously exists. 

Tangible interfaces can
for example
significantly improve 
spatial performance over graphical interfaces 
for tasks like matching 2D and 3D representations \cite{Cuendet2012}.








% Mental rotation
Mental rotation 
-- a cognitive task commonly used in psychometric tests of spatial ability -- 
can be physically simulated 
simply by grasping the object and rotating it (Fig.~\ref{fig:physical_rotation}).
%
While tasks like spatial visualization and mental rotation are used assess spatial ability
\cite{Uttal2013a,Uttal2013,Ormand2014}, 
space is not just visualized, but also felt.
Space need not be imagined to be transformed. 
%--haptic feedback about space informs subconscious pragmatic representations that rapidly generate action \cite{Jeannerod1997}. 
Mental rotation and spatial visualization, therefore, 
may tell us very little about embodied spatial ability. 







LEE 2009

—spatial visualization and spatial orientation—or whether a third—spatial
relations—is also a fundamental dimension of spatial ability (Lohman, 1979; Gilmartin &
Patton, 1984; Golledge & Stimson, 1997; Montello et al., 1999). Spatial visualization is
the ability to mentally manipulate, rotate, twist or invert pictorially presented visual
stimuli. Spatial orientation involves the comprehension of the arrangement of elements
within a visual stimulus pattern, the aptitude for remaining unconfused by the changing
orientations in which a configuration may be presented and the ability to determine spatial
relations in which the body orientation of the observer is an essential part of the problem
184 J. Lee & R. Bednarz
(McGee, 1979). The dimension that is the least clearly defined, and the most contentious,
is spatial relations.
Spatial relations include abilities to recognize spatial distributions and spatial patterns,
to connect locations, to associate and correlate spatially distributed phenomena, to
comprehend and use spatial hierarchies, to regionalize, to orientate to real-world
frames of reference, to imagine maps from verbal descriptions, to sketch map, to
compare maps, and to overlay and dissolve maps. (Golledge & Stimson, 1997, p. 158)



LEE 2009

Spatial Skills Test
Psychometric tests have been used extensively to assess spatial abilities, especially spatial
visualization and spatial orientation (e.g. Kail et al., 1979; McGlone, 1981; Miller &
Santoni, 1986; Goldstein et al., 1990; Newcombe & Dubas, 1992). ‘Spatial’ in the
psychometric tests, however, refers to small-scale (table-top) space which is not the scale
most pertinent to geography. Furthermore, psychometric tests hardly assess the ‘spatial
relations’ dimension, which is most closely related to GIS activities (Self et al., 1992;
Golledge, 1993; Lee & Bednarz, 2004). Montello et al. and Golledge argued that:
[T]he restricted definition of spatial ability, as incorporated into many psychometric
tests, contrasts with the richness of the general literature on spatial activities and
spatial behavior, much of it from disciplines other than psychology. (Montello et al.,
1999, p. 517)
[Spatial knowledge] includes higher level concepts such as hierarchy, surface,
association, connectivity, pattern, and so on ... Psychologists have neglected height
Effect of GIS Learning on Spatial Thinking 185
or relief in their examination of spatial phenomena. In contrast, the geographer
commonly represents spatial interactions, movements, or even the basic distribution
or pattern of phenomena as surfaces. They are sometimes represented in two
dimensional form and sometimes represented as three dimensional surfaces.
(Golledge, 1993, p. 28)
Considering that GIS has been developed to deal with spatial information at the
geographic scale, psychometric tests suffer obvious limitations for measuring the effects
of GIS learning on spatial ability (Golledge, 1993; Mark & Freundschuh, 1995).



WAKABAYASHI 2011 !!!!!!!

Spatial abilities are cognitive skills fundamental to spatial thinking, being composed of spatial
visualization, spatial orientation, and spatial relation [5]
[GOLLEDGE 1997)

. A variety of psychological studies have
developed and used tests for measuring spatial abilities [6,7]. Since most of them are paper-and-pencil
tests in small-scale space, they are not necessarily applicable to spatial thinking in large-scale geographic
space. Hegarty and Waller [7], who reviewed previous studies on spatial abilities, found very little
correlation between large- and small-scale spatial abilities. 



