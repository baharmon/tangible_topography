% !TEX encoding = UTF-8 Unicode

% Compilation using 'acmsmall.cls' - version 1.3 (March 2012), Aptara Inc.
% (c) 2010 Association for Computing Machinery (ACM)
%
% Questions/Suggestions/Feedback should be addressed to => "acmtexsupport@aptaracorp.com".
% Users can also go through the FAQs available on the journal's submission webpage.
%
% Steps to compile: latex, bibtex, latex, latex


\documentclass[prodmode,acmtochi]{acmsmall} % Aptara syntax

% Package to generate and customize Algorithm as per ACM style
\usepackage[ruled]{algorithm2e}
\renewcommand{\algorithmcfname}{ALGORITHM}
\SetAlFnt{\small}
\SetAlCapFnt{\small}
\SetAlCapNameFnt{\small}
\SetAlCapHSkip{0pt}
\IncMargin{-\parindent}

% Packages
\usepackage[super]{nth}
\usepackage[inline]{enumitem}
\usepackage{moreenum}
\usepackage{tabu}
\usepackage{booktabs}
\usepackage{array}
\newcommand{\ra}[1]{\renewcommand{\arraystretch}{#1}}
\newcommand{\urlhttp}[1]{\href{http://#1}{\nolinkurl{#1}}}
\newcommand{\urlhttps}[1]{\href{https://#1}{\nolinkurl{#1}}}

% Metadata Information
\acmVolume{9}
\acmNumber{4}
\acmArticle{39}
\acmYear{2010}
\acmMonth{3}

% Copyright
%\setcopyright{acmcopyright}
%\setcopyright{acmlicensed}
%\setcopyright{rightsretained}
%\setcopyright{usgov}
%\setcopyright{usgovmixed}
%\setcopyright{cagov}
%\setcopyright{cagovmixed}

% DOI
\doi{0000001.0000001}

%ISSN
\issn{1234-56789}

% Document starts
\begin{document}

% Page heads
\markboth{B. Harmon et al.}{Embodied Spatial Cognition in Tangible Computing}

% Title portion
\title{Embodied Spatial Cognition in Tangible Computing} % Spatial Performance in Tangible Computing 
\author{BRENDAN ALEXANDER HARMON
\affil{North Carolina State University}
ANNA PETRASOVA
\affil{North Carolina State University}
VACLAV PETRAS
\affil{North Carolina State University}
HELENA MITASOVA
\affil{North Carolina State University}
ROSS ​KENDALL MEENTEMEYER
\affil{North Carolina State University}
EUGENE BRESSLER
\affil{North Carolina State University}
ART RICE
\affil{North Carolina State University}}

\begin{abstract}
%
\end{abstract}

%
% The code below should be generated by the tool at
% http://dl.acm.org/ccs.cfm
% Please copy and paste the code instead of the example below. 
%
\begin{CCSXML}
<ccs2012>
<concept>
<concept_id>10003120.10003121</concept_id>
<concept_desc>Human-centered computing~Human computer interaction (HCI)</concept_desc>
<concept_significance>500</concept_significance>
</concept>
<concept>
<concept_id>10003120.10003121.10003122.10011749</concept_id>
<concept_desc>Human-centered computing~Laboratory experiments</concept_desc>
<concept_significance>500</concept_significance>
</concept>
</ccs2012>
\end{CCSXML}

\ccsdesc[500]{Human-centered computing~Human computer interaction (HCI)}
\ccsdesc[500]{Human-centered computing~Laboratory experiments}
%
% End generated code
%

\keywords{Human-computer interaction, tangible interfaces, interaction design, physical computation, embodied cognition, spatial thinking, geospatial modeling}

\acmformat{Brendan A. Harmon, Anna Petrasova, Vaclav Petras, Helena Mitasova, Ross K. Meentemeyer, Eugene H. Bressler, and Art Rice, 2016. Embodied Spatial Cognition in Tangible Computing.}

\begin{bottomstuff}
Author's addresses: B. A. Harmon {and} A. Petrasova {and} V. Petras {and} H. Mitasova {and} R. K. Meentemeyer, Center for Geospatial Analytics, North Carolina State University; B. A. Harmon, E. H. Bressler {and} A. Rice, Department of Landscape Architecture, North Carolina State University.
\end{bottomstuff}


%\maketitle



% difference 
In order to compare participants' modeling performance between sets 
we computed the difference 
between the linearly regressed reference elevation and 
the mean elevation for each set.
%
The difference between the reference and mean elevation maps should show
where the mean elevation values for each set are too low or too high. 
%
There were, however, systematic errors in the scanned models.
%
Table \ref{table:scatterplots} shows the vertical shift 
in the hand sculpted and projection augmented models 
caused by scanning and georeferencing.
%
We used linear regression  
to account for these systematic errors 
in the difference calculation, 

\begin{equation}
\label{eq:regressed_difference}
\Delta = (a + b * z_0) - \overline{z}
\end{equation}

where:

\hspace*{1em} $\Delta$ is the difference

\hspace*{1em} $z_0$ is the reference elevation map

\hspace*{1em} $\overline{z}$ is the mean elevation of maps in a set

\hspace*{1em} $a$ is the intercept~/~offset of the regression line

\hspace*{1em} $b$ is the gain~/~slope of the regression line.\\

We vertically rescaled and translated the reference elevation 
using the linear regression of the reference and mean elevation maps
calculated with the module \textit{r.regression.line} \cite{r.regression.line}.
%
Then we calculated the difference between 
the linearly regressed reference elevation 
and the mean elevation of all maps in a set
with the module \textit{r.mapcalc} \cite{r.mapcalc}. 

\pagebreak

In order to compare participants' modeling performance between sets 
we computed the difference 
between the linearly regressed reference elevation and 
the mean elevation for each set.
%
The difference between the reference and mean elevation maps should show
where the mean elevation values for each set are too low or too high. 
%
There were, however, systematic errors in the scanned models.
%
Table \ref{table:scatterplots} shows the vertical shift 
in the hand sculpted and projection augmented models 
caused by scanning and georeferencing.
%
We used linear regression 
%the linear regression of the reference and mean elevation maps
%calculated with the module \textit{r.regression.line} \cite{r.regression.line}
to vertically rescale and translated the reference elevation 
in order to account for these systematic errors
in the difference calculation,

\begin{equation}
\label{eq:regressed_difference}
\Delta = (a + b * x) - y
\end{equation}

where:

\hspace*{1em} $\Delta$ is the difference %[ft]

\hspace*{1em} $x$ is the reference elevation map %[ft]

\hspace*{1em} $y$ is the mean elevation of maps in a set %[ft]

\hspace*{1em} $a$ is the offset

\hspace*{1em} $b$ is the slope.\\








%\begin{equation}
%\label{eq:difference}
%z = x - y
%\end{equation}
%
%where:
%
%\hspace*{1em} $x$ is the reference elevation map %[ft]
%
%\hspace*{1em} $y$ is the mean elevation of all maps in a set %[ft]
%
%\hspace*{1em} $z$ is the difference. %[ft]
%
%%--------------------------------------
%
%\begin{equation}
%\label{eq:linear_regression}
%y = a + b*x
%\end{equation}
%
%where:
%
%\hspace*{1em} $x$ is the reference elevation map %[ft]
%
%\hspace*{1em} $y$ is the mean elevation of all maps in a set %[ft]
%
%\hspace*{1em} $a$ is the offset
%
%\hspace*{1em} $b$ is the slope.
%
%%--------------------------------------
%
%\begin{equation}
%\label{eq:regressed_difference}
%z = (a + b * x) - y
%\end{equation}
%
%where:
%
%\hspace*{1em} $x$ is the reference elevation map %[ft]
%
%\hspace*{1em} $y$ is the mean elevation of all maps in a set %[ft]
%
%\hspace*{1em} $z$ is the difference %[ft]
%
%\hspace*{1em} $a$ is the offset
%
%\hspace*{1em} $b$ is the slope.
%
%\pagebreak

%def difference(real_elev, scanned_elev, new, env):
%    """!Computes difference of original and scanned (scan - orig)."""
%    regression='regression'
   
%    regression_params = gcore.parse_command('r.regression.line', flags='g', mapx=scanned_elev, mapy=real_elev, env=env)

%    gcore.run_command('r.mapcalc', expression='{regression} = {a} + {b} * {before}'.format(a=regression_params['a'], b=regression_params['b'], before=scanned_elev, regression=regression), env=env)

%    gcore.run_command('r.mapcalc', expression='{difference} = {regression} - {after}'.format(regression=regression, after=real_elev, difference=new), env=env)
%    gcore.run_command('r.colors', map=new, color='differences', env=env)


%--------------------------------------
%        regression_params = gscript.parse_command('r.regression.line',
%            flags='g',
%            mapx=mean_dem_before,
%            mapy=mean_dem_after,
%            overwrite=overwrite)
%--------------------------------------
%        gscript.run_command('r.mapcalc',
%            expression='{regression} = {a} + {b} * {before}'.format(a=regression_params['a'],
%            b=regression_params['b'],
%            before=mean_dem_before,
%            regression=mean_dem_regression),
%            overwrite=overwrite)
%--------------------------------------
%        gscript.run_command('r.mapcalc',
%            expression='{difference} = {regression} - {after}'.format(regression=mean_dem_regression,
%                after=mean_dem_after,
%                difference=mean_dem_regression_difference),
%            overwrite=overwrite)







\end{document}



