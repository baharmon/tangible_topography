

%% tangibles can enhance spatial performance
%Through a series of experiments
%we found that tangible interfaces for spatial modeling
%can enhance 3D spatial performance 
%in terms of speed, accuracy, and process. 
%% coupling
%By comparing digital, analog, and projection-augmented modeling 
%we found that coupling digital and physical models as a tangible interface 
%can combine the affordances of digital and analog tools
%-- enabling an embodied modeling process enriched with digital data --
%so that users can model intuitively, quickly, and precisely. 
%Even 3D modeling experts 
% performed better with the tangible interface 
%-- building more accurate models 
%that better represented the morphology of the landscape --
% because they could work faster
% creating and refining details sooner.
%% transfer
%They were able to transfer and effectively use the
%spatial skills and abilities they had developed through digital modeling
%with the tangible interface.
%% analytics
%We also found that tangible interaction with real-time geospatial analytics
%can encourage iterative modeling processes.
%With the real-time difference analytic and water flow simulation  
%users worked in rapid cycles of 
%sculpting and digitally informed critical analysis
%to build accurate models that
%correctly represented the topographic and hydrologic morphology.
%% process
%Through this embodied process of reflection-in-action 
%users were able to
%observe spatial patterns, forms, and processes, 
%generate and test hypotheses, 
%and draw inferences. 
%% cog offloading
%The experiments showed that users 
%were able to offload enough of the cognitive work 
%of sensing and manipulating space
%onto their bodies
%that they could understand the
%computational analytics
%and adaptively re-strategize.
%% future research
%Further experiments are needed
%to explore the role of 
%spatial cognition, affect, motivation, and metacognition 
%in tangible modeling.
%
%% summary
%The experiments show that Tangible Landscape,
%a tangible interface for GIS, 
%works as theorized and designed -- 
%coupling a physical and digital model of a landscape
%enables users to 
%cognitively grasp topography,
%intuitively shape and interact with multidimensional space, 
%and offload enough cognitive work to understand 
%real-time geospatial analytics. 
%% process 
%With Tangible Landscape users can intuitively interact with 
%spatial data and scientific models using their bodies. 
%% users
%While novices should be able to effectively learn about 
%multidimensional space and
%rapidly improve their spatial abilities 
%with Tangible Landscape, 
%experts can effectively use it to 
%rapidly develop, prototype, and test 
%hypotheses about space and spatiotemporal processes.






