\label{appendix:methodology}

% ---------------------------- TOPOGRAPHIC  ----------------------------
\section{Topographic modeling results}
\label{appendix:topographic_results}

% ALL PARTICIPANTS
\paragraph{All participants}
Tables \ref{table:coupling_experiment},
\ref{table:percent_cells}, 
and \ref{table:distance}
show the results for all participants. 
%
% DIGITAL
When digitally modeling with Rhino
participants were relatively 
consistent and accurate except in the interior space.
The interior space -- the main valley and the ridge -- 
tended to have serious errors.
%
They built very abstract massing models 
approximating the general shape of the landscape
without any detail. 
%
The slopes in these models 
were consistently too low and gradual.
They did not have enough curvature 
to create distinct landforms. 
%
These models only hinted at 
the morphology of landscape
with a small cluster of valley cells
and another small cluster of ridge cells.  
Key features like the stream channel
were not represented. 

% digital observations
We observed that most participants had trouble
judging depth and perceiving interior space 
when digitally modeling. 
We also observed most participants 
using similar digital modeling strategies 
in a relatively linear fashion.
Since the reference data -- the 3D contour curves -- 
clearly represented profiles, 
many participants first modeled
the borders of the landscape
relying heavily
on the front and side viewports.
Then in perspective view 
they began to pull up relative high points 
to build a rough massing model of the topography.
Finally they began to refine its shape 
by pulling down relative low points to 
steepen slopes and form valleys.
%
The 3D modeling experts, however, 
used unique modeling strategies
developing their own techniques as they worked.
One worked exclusively in the perspective viewport
continually orbited above and below the model 
in order to compare the top and the bottom
as he pushed and pulled points
in a very freeform, iterative process. 
%
Only the experts had time to rebuild the 
10 x 10 grid of control points as a 20 x 20 grid. 
This meant that all of the other models
had half the effective resolution
and were thus more approximate
representations of the landscape. 

% ANALOG
When sculpting by hand 
participants used very different modeling strategies 
and as a result they built very different models
as shown by the standard deviation of elevation.
While inconsistent, 
their models were relatively accurate
and captured the key landforms.
While they tended to over exaggerate the main ridge, 
they did represent
this ridge and most of the central valley.
Due to slumping sand
these models tended to be too low along the edges. 

% analog observations
We observed participants using 
a wide range of different modeling strategies
and techniques when sculpting by hand. 
They tended to work in a very freeform manner 
-- switching freely between
adding, removing, pressing, pushing, pulling, or smoothing
sand. 
% tool use
Some used only their bare hands, 
some used only the wooden sculpting tool,
and others used their bare hands to sculpt 
and the tool the refine details. 
% 3d scale use
Most participants used the 3D scale
to build highest point at the right height
-- some even buried the 3D scale in their model
building around it. 
%
Some participants instead 
picked up the reference model
and used its edges -- its profiles -- to build 
the sides of the model. 
Some participants also ran their fingers 
over the reference model to feel its shape.
A few  participants, however,
ignored both the 3D scale
and the reference model. 

We observed participants 
freely and rapidly switch between 
sculpting the sand with 
their whole hands,
their palms,
the blade of their hands, 
or just their fingertips 
depending on whether 
they wanted to make a big move
or a fine-tuned refinement.
%
While participants could select and transform
a group of control points in Rhinoceros,
their control was limited by the grid size, i.e. the number of points,
and their speed by the need to continually change the selection.
%
When we tested Vue as an alternative
with 3D painting and sculpting tools
with parameters like size and intensity
we found that these parameters 
required continually tuning,
which interrupted the modeling process 
and slowed down interaction.

% interviews
See Table \ref{table:interviews} for select comments from interviews.
One participant said that she felt anxious when digitally modeling,
but felt calmer and more relaxed when hand sculpting sand. 
She found hand sculpting to be more intuitive, saying that
while digital modeling had `a long learning curve,' 
she could read the sand model `like braille\ldots 
I could feel the shape with my fingers.'

% AUGMENTED
Overall participants performed best
with projection-augmented modeling
building more accurate models 
that represented the major landforms.
%
These models had the lowest mean minimum distance 
for cells with concentrated water flow, ridges, and valleys
(see Tables \ref{table:percent_cells} \& \ref{table:distance}). 
%
We observed participants using the same
modeling techniques, strategies, and tool use
as they did when sculpting by hand. 
They worked in the same freeform manner, but 
-- with the aid of the projected contours and elevation --
had more consistent results 
as shown by the standard deviation of elevations.
%
Overall, they had more accurate results
-- albeit with systemic errors along the borders 
due to slumping sand --
as shown by the mean difference and
standard deviation of differences. 
%
They correctly represented the main ridge and 
the central valley, but tended to miss details
like the y-shaped branch at the head of
of the stream channel. 

One participant found that 
with projection-augmented modeling 
he was able to effectively combine 
the affordances of hand sculpture 
with the extra layer of data. 
%
He described an iterative strategy of additive modeling
in which the projected `contours were just a guide' --  
`My general strategy was additive. 
I felt with my hands to try to match the contours. 
If I saw concavity in the contours 
then I felt the sand and sculpted that concavity.
Finding the relative height, however, was challenging -- 
it was subtle.'

\paragraph{Students vs. academics and professionals}
%
Table \ref{table:students} and Table \ref{table:professionals}
compare students' results 
with academics and professionals. 
% overview
Overall the academics and professionals
only performed slightly better than the students.
% digital
When digitally modeling both groups 
built very approximate massing models
that only hinted at the landforms. 
While the professionals and academics 
managed to create a small cluster of valleys cells
in the central stream channel, 
the students tended to miss this feature. 
% analog
When sculpting by hand
both groups' performance improved dramatically;
they built more accurate models
as shown by the standard deviation of difference
and better represented the landforms
roughly capturing the main ridge and valley. 
%
With projection augmented modeling
both groups' performance increased even more
as shown by the low standard deviation of difference.

\paragraph{Landscape architecture vs. GIS students}
%
Table \ref{table:landscape_students} and
Table \ref{table:gis_students}
compare landscape architecture and GIS students' results.
%
% overview
The GIS students built more abstract, approximate models
than the landscape architecture students who tended to 
over-exaggerate the shape of the landscape.
%
With each technology the GIS students' 
performance improved;
they built more and more accurate models 
with more distinct landforms.
The landscape architecture students' performance, however, 
did not improve significantly between
analog and projection-augmented modeling. 
% digital 
With digital modeling both groups
had major errors along the ridge,
but the GIS students also had serious problems
with the interior space.
The GIS students missed
the central valley entirely, while the landscape students
hinted at it with a small cluster of valley cells. 
%
When hand sculpting 
both groups made major improvements.
The GIS students began to form
the main ridge and valley. 
They tended to build too large a ridge
with the highest point near its tip
and significant errors on its slopes. 
The landscape architecture students tended to 
hand sculpt distorted, exaggerated landscapes 
with extras landforms, 
but roughly captured the y-shaped stream. 
%
With projection augmented modeling 
the GIS students captured
the central valley and ridge,
correctly modeled steeper slopes,
and had much less errors as shown by 
the standard deviation of difference.
%
The landscape architecture students, however,
did not make significant improvements 
over their hand sculpted models. 

\paragraph{Academics and professionals with and without 3D modeling expertise}
%
Table \ref{table:3d_novices} and Table \ref{table:3d_experts}
compare landscape architecture academics and professionals 
with and without 3D modeling expertise. 
% overview
While the other academics and professionals
performed poorly with digital modeling, 
better with analog modeling, 
and best with projection augmented modeling,
the expert 3D modelers built very accurate models
with each technology, but represented the landforms
best with projection augmented modeling.
% digital
While the rest of the academics and professionals 
had major errors on the ridge 
in the digital modeling task,
the expert 3D modelers built accurate models 
with very low mean difference and 
standard deviation of difference. 
The 3D experts worked faster
and had time to rebuild denser grids of control points
so that they could build more detailed,
higher resolution models.
Despite their accuracy their digitally sculpted models 
still only hinted at the landforms 
with small clusters of valley and ridge cells. 
% analog
The 3D modeling experts represented 
more complete landforms when sculpting by hand,
capturing details like the y-shaped branch of the stream,
but missed much of the ridge. 
% augmented
With projection augmented modeling 
the expert 3D modelers performed even better
successfully capturing
the ridge, the valley, and its y-shaped branch
with few anomalous features. 
%
They used the wooden modeling tool 
to cut clean edges around their models
minimizing systematic errors caused by slumping sand.
