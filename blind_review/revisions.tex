- tangible interaction is widely studied in the HCI community but the paper makes no references to findings here nor does it use a method developed and tested in the HCI field for its own purpose.

We have revised the 'Research objectives' section, adding a review of relevant case studies and user studies to explain why we developed a new method for assessment. 

- in research objectives you ask LA to "perform basic design tasks" what are these tasks and could by listing them later better be evaluated for which tasks the technology caters best? Thus the objectives stay vague which makes it hard to evaluate and position the findings at the end.

We have revised the 'Research objectives' section to address this comment. The 'basic design tasks,' however, were already outlined in this research objectives ("...basic landscape design tasks -- topographic modeling, cut-and-fill analysis, and water flow modeling --...") and were described in detail in the following 'Topographic modeling' section, 'Cut-and-fill analysis' section, and 'Water flow modeling' section. 
We rewrote the aim and objectives to help clarify the basic design tasks. 

- Further the method is not to well defined other then "18 participants tried to accurately model the landscape" as mentioned prior it would have been helpful here to quote or reference a or various existing methods our of UX / UI design to validate ones own method.

As noted above, we have added references to existing studies and methods on page 2 in the 'Research objectives' section. The method, however, is well defined. We describe it in detail on pages 3-8 in the 'Methods' subsections. There are also video links demonstrating the tasks ( https://youtu.be/dSyrHAuu698 , https://youtu.be/vA1xwMSaGV4 , https://youtu.be/STYHUHNaWdY , https://youtu.be/1uEvzMJWh_E , https://youtu.be/Q3elMIRCYSk , & https://youtu.be/61hsXgb3MLY ). Furthermore the supplemental materials have links to detailed descriptions of the experimental procedure ( https://osf.io/82gst/ )

- Again in HCI 18 participants would be considered as 'too thin' to get results.

We used a small group of participants because the tasks were very time intensive. 
Yes, more participants would have been better. 
The results, however, show clear, significant morphological differences.
We think the spatially-explicit, morphological aspect of the study is original, highly relevant, and worth the relatively small sample size. 

- Not having semi structured interviews is a pity, could one not ask the academics? In table 11 interviews are mentioned, what were the questions as these are most likely related to the objectives of the research which have been in my opinion not well be defined.

I have added the semi-structured interview guidelines to the supplemental materials. The supplemental materials also includes links to online repositories (ie. the Open Science Framework) with the experimental procedure, the interview guidelines, code, data, and results. The links were omitted from the paper for double blind review. Here is the link for the Open Science Framework repository: https://osf.io/82gst/

Conclusion:
The journal article needs a significant part in where the methods of testing tangible interfaces are explained, here the authors can refer to work by others in the HCI field and use their framework and adopted it to make it their own. Further the objectives of the research should be clearly stated and all three experiments first introduced and argued for with the objectives. As the article tends towards practical and user testing a larger user group (ie. practitioners) would also help.

As noted above we have added a review of relevant studies and methods to the 'Research objectives' section. And we have refined the aim and objectives.

Unfortunately we can not continue the experiment and test more users. In the future we hope to run a much larger, better designed study about the next generation of this technology, addressing something like 3D planting design with 3D real-time rendering.
