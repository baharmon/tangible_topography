%- in research objectives you ask LA to "perform basic design tasks" what are these tasks and could by listing them later better be evaluated for which tasks the technology caters best? Thus the objectives stay vague which makes it hard to evaluate and position the findings at the end.

% -------------ORIGINAL----------------------

\subsection{Research objectives}
Many of the theoretical underpinnings of tangibles 
remain unproven and unexplored. 
Do current approaches to tangible, embodied interfaces
really work as theorized? 
Can designers really model more intuitively and 
design more effectively with tangibles?
And how do tangibles mediate the design process?
% experiments
In order to assess the effectiveness 
of tangible modeling for landscape architects 
we conducted a series of user studies
assessing how well participants performed
basic landscape design tasks 
-- topographic modeling, cut-and-fill analysis, and water flow modeling --
using Tangible Landscape.\\

Our research objectives were to:
%
\begin{itemize}
% performance
\item Assess how well landscape architects could perform 
basic design tasks using a tangible interface for landscape modeling
% process
\item Study how tangible interaction mediated
landscape architects' design process
\end{itemize}

% -----------------REVISED------------------------
The aim of this research was to test whether
sandbox-style tangibles could be an effective tool
for landscape design. 
%
%Our research objectives were 
%to assess how well landscape architects could:
%\begin{enumerate}
%\item model topography, 
%\item analyze cut-and-fill, 
%\item 
%\end{itemize}
%using a sandbox-style tangible interface for landscape modeling.
%
Our research objectives were 
to assess how well landscape architects could:
\begin{enumerate*}[label=\alph*),font=\itshape]
\item model topography, 
\item analyze cut-and-fill, 
\item and model water flow
\end{enumerate*}
using a sandbox-style tangible interface for landscape modeling.
%
In order to assess the effectiveness 
of tangible modeling for landscape architects 
we conducted a series of user studies
assessing how well participants performed
these basic landscape design tasks 
%-- topographic modeling, cut-and-fill analysis, and water flow modeling --
using Tangible Landscape.

There have been very few 
case studies, \cite{Ishii2002,Tateosian2010,Petrasova2015}
qualitative user studies, 
\cite{Shamonsky2003,Woods2016}
and quantitative studies \cite{Schmidt-daly2016b}
about sandbox-style tangibles.
%
%The quantitative study by
%Schmidt-Daly et al. \cite{Schmidt-daly2016b} 
%does not address the unique, most important,
%and hardest to assess affordance of 
%sandbox-style tangibles --
%freeform 3D sculpting. 
%
While the quantitative study by
Schmidt-Daly et al.~\cite{Schmidt-daly2016b} 
assessed both learning and task performance,
it did not assess the unique and most important 
affordance of sandbox-style tangibles --
freeform 3D sculpting. 
% 
Sandbox-stye tangibles uniquely afford
the ability to sculpt digitally augmented, 
freeform 3D volumes
with ones bare hands. 
%
While there are standard, validated
pre- and post-tests for assessing learning
in human-computer interaction,
methods for assessing task performance 
are specific to the task
(see for example \cite{Cuendet2012}).
%
Furthermore,
in a pilot study we found that the only 
validated test for reading topography -- the
Topographic Map Assessment test
\cite{Newcombe2015} --
was too easy for novice landscape architects.
%; they aced the pre-test. 
%
Therefore, we developed 
new, spatially-explicit methods
for assessing performance
in freeform 3D modeling tasks.
%
We used raster statistics, 
morphometric analyses, and geospatial simulation
to assess the spatial accuracy, pattern and distribution
of the results for each task. 
